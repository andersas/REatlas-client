\subsubsection{\_select\_current\_file()}


\begin{verbatim}
 _select_current_file(filename,username="",intent="")

               filename: Filename to select.
               username: If given, select a file in that users directory
                         instead of your own.
               intent: If given and equal to "download",
                       return an error if the file does not exist.

          Selects which binary file to send to or receive from the server.
          These are send as raw binary instead of through JSON-RPC.

          
\end{verbatim}
\subsubsection{delete\_file()}


\begin{verbatim}
 delete_file(filename,username="")

               filename: Name of file to delete.
               username: If given, delete the file in this users directory instead of your own.
          
          Example:
               delete_file(filename="foo.npy",username="someguy")

          Delete foo.npy under someguys folder.

          Returns True on success.
          
\end{verbatim}
\subsubsection{list\_files()}


\begin{verbatim}
 list_files(username=""):

          Return a list of your files and their sizes in bytes.

          Arguments:
               username: If given, list this users files instead of your own.

          Example:
               list_files()
          returns a list of all your files on the server.

          
\end{verbatim}
\subsubsection{login()}


\begin{verbatim}
 Log in to the server with supplied credentials. Returns True
          on success, and False on invalid credentials.
          
          Keyword arguments:
               username
               password
          
          It is a good idea to login immediately after a connection is made,
          as the server times out very quickly if no user is logged in.
\end{verbatim}
\subsubsection{job\_priority()}


\begin{verbatim}
 Set or get current priority level.
          
          job_priority(priority=None) -> int

          Sets priority level of current session to priority.
          Any jobs submitted afterwards will have this priority.
          Priority must be 0, 1 or 2. The higher the number, the higher the priority.
          If not specified, only return current priority level.

          Note: If priority exceeds the users max_priority setting,
          this call sets the priority level to max_priority.
          
          
          Examples:
               job_priority()
               1

          Your jobs will be submitted with job priority 1.

               job_priority(priority=2)
               1
          You tried to set a priority level of 2, but your maximum allowed
          priority is 1. 
          
          
\end{verbatim}
\subsubsection{notify\_by\_mail()}


\begin{verbatim}
 Gets/sets notify option of current session
         
          notify_by_mail(s,notify=None) -> Bool

          Notify must be None, False or True. If None,
          only return current notification setting.
          If True or False, set current notification setting
          to True or False, respectively.

          If the notification setting is True, jobs you create will send
          you a notification email when done. 
          
          Example:
               notify_by_mail(notify=False)
               False
          Jobs you submit during this session will no longer send an
          email to you when done.
          
          
\end{verbatim}
\subsubsection{\_get\_available\_methods()}


\begin{verbatim}
 Return a list of JSON-RPC methods (functions) that
          the server accepts. 
\end{verbatim}
\subsubsection{\_get\_method\_docstrings()}


\begin{verbatim}
 Returns a dictionary of Python docstrings for the
          methods provided by the server. 
\end{verbatim}
\subsubsection{\_get\_server\_endianness()}


\begin{verbatim}
 Gets the byte order used by the server CPU 
          
          _get_server_endianness() -> str

          Returns:
               'little' if the server is a little endian machine or
               'big' if the server is a big endian machine.
\end{verbatim}
\subsubsection{echo()}


\begin{verbatim}
 echo(message=""):
               returns message. 
\end{verbatim}
\subsubsection{generate\_error()}


\begin{verbatim}
 Generates an example error. 
\end{verbatim}
\subsubsection{list\_cutouts()}


\begin{verbatim}
 Returns a list of tuples with cutouts and used space.
          
               list_cutouts(all_users=False,loaded=False)

          If all_users is True, then list cutouts of all users
          instead of only the current user.

          If loaded is True, then list the cutouts loaded in the ramdisk
          instead of those on the hard disk.
          This requires super user privileges.

          Each cutout is given as user/cutout_name. Entries containing
          just user without /cutout_name gives the total space used by that users cutouts. 
\end{verbatim}

\subsubsection{submit\_dummy\_job()}


\begin{verbatim}
 
          Queue a job on the server that does nothing for 15 seconds.
          No arguments.
          
\end{verbatim}
\subsubsection{get\_queued\_job\_time()}


\begin{verbatim}

          Returns the estimated time (in seconds) of work currently
          in the queue by user name as a dictionary. 

          If you're not privileged, it only returns the estimated time
          your jobs account for and the total time. 
          
          Example:
               get_queued_job_time()
               {u'TOTAL': 3600,
                u'jens': 180}
          If your username is jens, you have jobs worth of 180 seconds in the queue.
          There is a total of 3600 seconds of jobs currently waiting.
          
          
\end{verbatim}
\subsubsection{get\_estimated\_time\_before\_completion\_of\_jobs()}


\begin{verbatim}
 Get estimated time for a specific job to be done.
          
          Considering all queued jobs, return the estimated
          time (in seconds) before a specific job completes.

          Arguments (optional):
               job_id: ID of the job to get an estimate for.
                       If not given, return a list of time left for all jobs.

          If job_id is specified and no job has that job id
          (e.g. if it is finished), then this returns None.

          Note: time estimates are rather rough.

          Example:
               timings = get_estimated_time_before_completion_of_jobs()
               timings[33]
               500

          Here you see that job 33 is estimated to be done in 500 seconds.
          If this is larger than the time estimate for job 33, this means
          that the job is waiting for other jobs to finish.
          
\end{verbatim}
\subsubsection{list\_queued\_jobs()}


\begin{verbatim}
 Get a list of jobs in queue and some info.

          No arguments.

          Returns a list of jobs.
          If you are not privileged, you will only see your own jobs.
          Each entry is a dictionary with the keys:
               user: Username of the job owner.
               name: Name of the job.
               job_id: ID of the job.
               time_estimate: Estimated job running time (in seconds).
               ETA: Estimated time left before job completion (in seconds).

          Note: time estimates are rather rough.
          
\end{verbatim}
\subsubsection{wait\_for\_job()}


\begin{verbatim}
 Waits (i.e. doesn't return before) job completes.
     
          Arguments:
               job_id: Id of the job to wait for.
               timeout: Wait for at most this many seconds.
                    Defaults to 3600 (1 hour).
          
          When the waiting is over, 
          the return value is True if the job is done, False otherwise. 
          
          Example:
               wait_for_job(job_id=33)
               [... nothing happens until job 33 is done,
                    or an hour has passed ...]
               False

          Indicates that the job is not done yet, and you should call
          wait_for_job again if you want to wait for it.
          
\end{verbatim}
\subsubsection{cancel\_jobs\_of\_user()}


\begin{verbatim}
 Cancel all jobs of a user.
          
          Without any arguments, this function cancels all your jobs
          that are waiting in the queue.

          Arguments:
               username: If given, cancel the specified users jobs instead. 
\end{verbatim}
\subsubsection{translate\_GPS\_coordinates\_to\_CFSR\_index()}


\begin{verbatim}
 Find the CFSR indices of given GPS coordinates.
          
          translate_GPS_coordinates_to_CFSR_index(latitudes,longitudes)
               -> [CFSRindex_0, CFSRindex_1]

          Arguments:
               latitudes and longitudes are lists of numbers corresponding
               to the requested GPS coordinates.
               latitudes[i] and longitudes[i] should be the GPS latitude
               and longitude of point i.

          Returns: A 2-element list of lists of first and second CFSR indices.
          I.e. [[latitude0,latitude1,...], [longitude0,longitude1,...]]. 
          
          Example:
               translate_GPS_coordinates_to_CFSR_index(latitudes =[55,56],
                                                       longitudes=[7 , 6])
          [[464, 467], [22, 19]]

          This shows that 55 degrees north, 7 degrees east has the CFSR
          index 464,22 and the point 56,6 has index 467,19
          
\end{verbatim}
\subsubsection{cutout\_CFSR\_rectangular\_by\_GPS\_coordinates()}


\begin{verbatim}
 Cut out a rectangle of the CFSR data based on GPS coordinates.
          
          Functionally identical to CFSR_rectangular_cutout_by_CFSR_indices,
          but southwest and northeast should be GPS coordinates instead of 
          CFSR indices. The first entry in southwest/northeast is the latitude
          and the second is the longitude.
         
          This method simply translates the GPS coordinates to
          CFSR indices via the translate_GPS_coordinates_to_CFSR_index method
          before calling CFSR_rectangular_cutout_by_CFSR_indices.
          
\end{verbatim}
\subsubsection{cutout\_CFSR\_rectangular\_by\_CFSR\_indices()}


\begin{verbatim}
 Cut out a rectangle of the CFSR data based on CFSR indices.

               CFSR_rectangular_cutout_by_CFSR_indices(name,southwest,northeast,
                    firstyear=None,lastyear=None,firstmonth=None,lastmonth=None,
                    double=False,nowind=False,nopv=False) -> int or None

               name: Name of the cutout
               southwest, northeast: lists of CFSR indices of the southwesternmost
                    and northeasternmost points in the rectangle.
                    These should be 2-element lists of integers.
               firstyear,firstmonth,lastyear,lastmonth: If specified,
                         these set the temporal range of the cutout.
                         E.g. if firstyear=2000, only use from 2000 and onwards.
                         If lastyear=2004, only use data before 2004.
               double: If True, dump the cutout in double precision. This requires
                       super user privileges.
               nowind: If True, don't cut out data needed for wind conversion
               nopv: If True, don't cut out data needed for photovoltaics conversion

              Returns the job id of the cutout job submitted to the queue,
              or None if the queue is closed or full.
                       
\end{verbatim}
\subsubsection{cutout\_CFSR\_individual\_points\_by\_GPS\_coordinates()}


\begin{verbatim}
 Cut out individual CFSR grid cells based on GPS coordinates.
          
          CFSR_individual_cutout_by_GPS_coordinates(name,latitudes,longitudes,
               firstyear=None,lastyear=None,firstmonth=None,lastmonth=None,
               double=False,nowind=False,nopv=False)
     
          Translate latitudes and longitudes to CFSR indices with
          translate_GPS_coordinates_to_CFSR_index() and call
          CFSR_individual_cutout_by_CFSR_indices() with the given arguments
          and the translated indices.
          
          
          Example:
               cutout_CFSR_individual_points_by_GPS_coordinates(name="test",
                    latitudes=[55,56],longitudes=[7.5,9])

               cuts out two points 55,7.5 and 56,9 in GPS coordinates.
          
          
\end{verbatim}
\subsubsection{cutout\_CFSR\_individual\_points\_by\_CFSR\_indices()}


\begin{verbatim}
 Cut out individual CFSR grid cells based on CFSR indices.

          CFSR_individual_cutout_by_CFSR_indices(s,name,
               first_indices,second_indices,
               firstyear=None,lastyear=None,firstmonth=None,lastmonth=None,
               double=False,nowind=False,nopv=False) -> int or None

               name: Name of the cutout
               first_indices,second_indices: Arrays of CFSR first and second indices.
                    The i'th point should have index first_indices[i],second_indices[i].
               firstyear,firstmonth,lastyear,lastmonth: If specified,
                         these set the temporal range of the cutout.
                         E.g. if firstyear=2000, only use from 2000 and onwards.
                         If lastyear=2004, only use data before 2004.
               double: If True, dump the cutout in double precision. This requires
                       super user privileges.
               nowind: If True, don't cut out data needed for wind conversion
               nopv: If True, don't cut out data needed for photovoltaics conversion

              Returns the job id of the cutout job submitted to the queue,
              or None if the queue is closed or full.  
              
              Example:
               cutout_CFSR_individual_points_by_CFSR_indices(name="test",
                    first_indices=[464, 467], second_indices=[22, 19]);

                    cuts out the CFSR points 464,22 and 467,19 and give
                    it the name "test".
              
\end{verbatim}
\subsubsection{delete\_cutout()}


\begin{verbatim}
 Delete a cutout
          
          delete_cutout(cutoutname,username=None)

          Delete cutout called cutoutname. If username is not specified, a cutout
          under the current user is deleted.
          If username (a string) is specified, delete that users cutout
          with that name.

          Deleting other users cutouts require super user privileges.
          
          Example:
               delete_cutout(cutoutname="test")

               deletes your cutout named test.
          
          
\end{verbatim}
\subsubsection{prepare\_cutout\_metadata()}


\begin{verbatim}
 Makes available all meta data for a given cutout in your server directory.
          
          Arguments:
               cutoutname: Name of the cutout to get data from.
               username: If given, find the cutout under this users cutouts
                         instead of your own.
          
          On successful return, a file named meta_<cutoutname>.npz will
          be available for download in your folder.

          It is a zipped file with the following numpy arrays:

               latitudes,longitudes: Contains latitudes and longitudes (in degrees)
                                     of each point in the cutout.
                                     The i,j'th entry corresponds to the i,j'th
                                     point in the cutout.

               dates: Timestamps of each hour in the cutout.
                      The i'th entry here corresponds to the i'th hour in the cutout. 

               onshoremap: A map of onshore points.
                           Cell i,j contains 1 if the point is onshore, 0 otherwise.

               heights: The height in of the grid cell.
                        Negative numbers indicate below sea level.
                        These values are derived by interpolating the
                        GEBCO 30 arc minute grid at the center of each grid cell.

          The .npz file can be opened directly by numpy's load() function.
          
\end{verbatim}
\subsubsection{\_get\_unique\_npy\_file()}


\begin{verbatim}

          Generate a file with a unique name in your directory.
          The filename will end with ".npy".
          
          This function could be useful if you do a lot of
          conversion + aggregation jobs for automatic name generation.

          Returns the name of the file, including the .npy extension. 
\end{verbatim}
\subsubsection{select\_cutout()}


\begin{verbatim}
 Selects a cutout for doing conversions.

          Arguments:
               cutoutname: Name of the cutout to select.
               username: If given, select this users cutout.

          All conversions will be done on the selected cutout.
          
\end{verbatim}
\subsubsection{get\_selected\_cutout()}


\begin{verbatim}
 Returns a tuple (cutoutname, username) for the currently selected cutout. 
\end{verbatim}
\subsubsection{convert\_and\_aggregate\_wind()}


\begin{verbatim}
 Submit a wind conversion and aggregation to job queue.

          Returns the job number of the conversion in the job queue.

          Arguments:
               result_name: Name of file for storing the result.
               onshorepowercurve: A powercurve object for use onshore.
               offshorepowercurve: A powercurve object for use offshore.
               capacitylayouts: A list of file names in your folder containing
                                capacitylayouts to apply.
                                These should all be numpy (.npy) files.
               save_sum: If true, store the sum of production of all grid cells
                         in result_name + "_sum.npy" in your folder on the server.
                         If save_sum is true, the list of capacity layouts can
                         be empty, in which case only the sum is calculated.
               onshoremap: If given, use this named .npy file in your server folder
                           as an onshoremap instead of the default CFSR land sea mask.
               nthreads: If given, run the conversion and aggregation
                         with this many threads. Requires privileges to set.
               worksize: If given, each thread converts/aggregates this many
                         hours at a time. Requires privileges to set.

     
          The result is stored in your folder on the server with the name
          you gave as parameter result_name. It is a numpy array with a number
          of timeseries equal to the number of layouts you specified.
          To get e.g. hour 100 for layout number 2, use result_name[99][1]. 

          A powercurve object is a dictionary with three keys: HUB_HEIGHT, V and POW.
          The value for HUB_HEIGHT is the hub height in meters above ground.
          The values for V and POW are equal length lists of numbers.
          POW[i] is the power output for a wind speed V[i].
          The V's must be increasing for the linear interpolation routine to work,
          and be in units of meters per second.

          Note: To get production values between 0 and 1, send scaled down power curves!
          The timeseries are simply a weighted sum over production in the entire cutout,
          with the layout cells as weights. The unit of the sum is the unit of the layout
          times the unit on the second axis of the power curve.
          
\end{verbatim}
\subsubsection{reset\_orientations()}


\begin{verbatim}
 Reset the list of solar panel orientation functions. 
\end{verbatim}
\subsubsection{add\_constant\_orientation\_function()}


\begin{verbatim}
 Add a constant orientation to the list of PV panel orientations
          for PV conversions. 
          
          Arguments:
               slope: The slope, i.e. angle between panel and ground, in degrees.
               azimuth: The east-west angle of the panel, in degrees.
                        Westward direction is positive, eastward negative.
               weight: Factor to apply to the result of the conversion
                       with this orientation.
          
\end{verbatim}
\subsubsection{add\_horizontal\_axis\_tracking\_orientation\_function()}


\begin{verbatim}
 Add a horizontal axis tracking orientation to the list of
          PV panel orientations for PV conversions. 
          
          Arguments:
               slope: The constant slope, i.e. angle between panel and ground, in degrees.
               weight: Factor to apply to the result of the conversion
                       with this orientation.
          
\end{verbatim}
\subsubsection{add\_vertical\_axis\_tracking\_orientation\_function()}


\begin{verbatim}
 Add a vertical axis tracking orientation to the list of
          PV panel orientations for PV conversions. 
          
          Arguments:
               azimuth: The constant east-west angle of the panel, in degrees.
                        Westward direction is positive, eastward negative.
               weight: Factor to apply to the result of the conversion
                       with this orientation.
          
\end{verbatim}
\subsubsection{add\_full\_tracking\_orientation\_function()}


\begin{verbatim}
 Add a full axis tracking orientation to the list of
          PV panel orientations for PV conversions. 
          
          Arguments:
               weight: Factor to apply to the result of the conversion
                       with this orientation.
          
\end{verbatim}
\subsubsection{convert\_and\_aggregate\_pv()}


\begin{verbatim}
 Submit a photovoltaics conversion and aggregation to job queue.

          Arguments:
               result_name: Name of file for storing the result.
               solar_panel_config: A solar panel configuration object,
               capacitylayouts: A list of names of files in your folder
                                containing capacitylayouts to apply.
                                These should all be numpy (.npy) files.
               save_sum: If true, store the sum of production of all grid cells
                         in result_name + "_sum.npy" in your folder on the server.
                         If save_sum is true, the list of capacity layouts can
                         be empty, in which case only the sum is calculated.
               
               nthreads: If given, run the conversion and aggregation
                         with this many threads. Requires privileges to set.
               worksize: If given, each thread converts/aggregates this many
                         hours at a time. Requires privileges to set.

          Returns the job number of the conversion in the job queue.
     
          The result is stored in your folder on the server with the name
          you gave as parameter result_name. It is a numpy array with a number
          of timeseries equal to the number of layouts you specified.
          To get e.g. hour 100 for layout number 2, use result_name[99][1]. 

          A solar_panel_config object is a dictionary with the following keys:
               A, B, C: The A,B, and C coefficients in the expression
                    reference_effeciency(I) = A + B(I) + C*log(I)
                    (I is the incident radiation). (See [1])
               D: The temperature power coefficient of the PV cell.
               NOCT: The Normal Operating Cell Temperature (in Kelvin)
               Tstd: The Standard Test Conditions temperature (in Kelvin)
               Tamb: The ambient temperature at STC (in Kelvin)
               Intc: Normal Testing Conditions irradiance (in W/m^2)
               ta: Product of transmittance and absorptance for the glass panel
                   in front of the solar cell. (0.9 is typical)
               threshold: Solar radiation below which the panel stops giving useful power
                          (set to e.g. 1 to avoid division by zero)
               inverter_efficiency: The (constant) efficiency of the inverter.
                                    The entire timeseries is multiplied with this number.

          The input orientations will be reset after a successful call.

          Note: To get production values in units of installed capacity
                (for a single orientation), calculate the A,B and C coefficients
                for 1 m^2 solar panel, and divide the weight by the rated production
                for 1 m^2 of the solar panel.

          The timeseries are simply a weighted sum over production in the entire cutout,
          with the layout cells as weights. The unit of the sum is the unit of the layout
          times the unit of the production.

          The production in each cell is calculated as a sum of the production
          for each orientation times that orientations weight.
          The unit of the production in each cell is proportional to W/m^2
          (depending on A,B and C) times the unit of the orientation weights.

          [1]:  "A robust model for the MPP performance of different types of PV-modules
          applied for the performance check of grid connected systems", 2004,
          by Hans Beyer, Gerd Heilscher and Stefan Bofinger
          
\end{verbatim}

