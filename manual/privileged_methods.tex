\subsubsection{shutdown()}


\begin{verbatim}

          Shut down the server. Takes no arguments. 
          Can only be executed by a privileged user. 
\end{verbatim}
\subsubsection{cancel\_all\_jobs()}


\begin{verbatim}
 Cancel all jobs in the queue. No arguments. Requires privileges. 
\end{verbatim}
\subsubsection{iterate\_mode()}


\begin{verbatim}
 iterate_mode(s) -> path as string
          
          The underlying weather data takes up large amounts of storage space.
          Therefore, only aggregated time series can be sent over the network.

          By entering iterate mode, you obtain an exclusive lock to the server.
          This allows you to extract conversions on a grid cell basis.
          If the server is idle for more than 10 minutes in iterate mode,
          it switches back to normal mode.
          
          For speed of calculation, the raw data is stored in a RAM disk
          on the server. The result arrays are also stored there.
          Whenever a new conversion is done, previous results are overwritten.
          The exclusive lock on the server is there to guarantee that
          this won't happen while you're reading your results.

          Because of limited network speed, the results are still not
          sent over the network. Therefore, you must have direct access
          to the server file system to obtain the results.

          Returns:

          On a successfull call to this method, a tuple containing first
          the path to the cutout in the RAM disk and second the path
          to your folder on the server is returned.
          You can use this to access the results/raw data and 
          to bypass the server download mechanisms and directly
          access your files.

          You can call this function multiple times even when already
          in iterate mode.
     
          A cutout is a folder with the following files:
          
               meta.json: JSON formatted dictionary with meta data (shape etc.)
                          Use this file first when opening the arrays.
                          If usefloat is false, all arrays are in
                          double precision. Else, influx.raw, outflux.raw,
                          result_pv.raw, result_wind.raw, roughness,
                          temperature.raw and wind-speed.raw are all stored
                          in single precision.
               
               dates: C array of 64 bit integers storing the UNIX timestamp
                      for each time step in the time series

               dates.npy: array of python datetime.datetime timestamps for
                          each time step in the time series.

               heights:   double array in the shape of the cutout where
                          each cell gives the height (in meters) in the
                          center of the grid cell

               influx.raw: Raw insolation data (W/m^2). Input for PV conversion.

               latitudes: double array in the shape of the cutout where
                          each cell gives the latitude (in degrees) of
                          the center of the grid cell.

               longitudes: ditto for longitudes

               onshoremap: 8 bit integer array in the shape of the cutout.
                           Contains 0 if the grid cell is offshore, and 1 
                           if it's onshore.

               outflux.raw: Raw reflected radiation data (W/m^2). This
                            is input data for PV conversion.

               result_pv.raw: Result of PV conversion.

               result_wind.raw: Result of wind conversion.

               roughness: 3D Array (monthly, not hourly!) of roughness
                          values for each grid cell.

               roughness_offsets: Array of 64 bit unsigned integers.
                                  The i'th entry gives the first hour 
                                  of the i'th month.
               
               temperature.raw: Raw input to PV conversion (Kelvin).
               
               wind-speed.raw: Raw input to wind conversion (m/s).


          The following example python program opens the wind result
          and prints out the grid cell with index 1,2 for hour 0.
          (counting from 0)

          import numpy
          import json

          with open("meta.json") as f: # Load meta information
               meta = json.load(f)

          dtype = numpy.double # Use single or double precision?
          if (meta["usefloat"]):
               dtype = numpy.single;
    
          shape = meta["shape"]
          num_hours = meta["length"]

          array_shape = tuple([num_hours] + shape); # Total shape of output

          # Open the result as a numpy array:
          wind_result = numpy.memmap("result_wind.raw",mode="r",dtype=dtype,
                                     shape = array_shape);

          print(wind_result[0][1][2]);


          Note that wind_result will change when a new wind conversion is run.
          This is because wind_result is a memory map of the result
          in the RAM disk.
          Therefore, you don't have to reopen the result_wind.raw file
          when a new conversion is done.
          
\end{verbatim}
\subsubsection{exit\_iterate\_mode()}


\begin{verbatim}
 Exit from iterate mode. 
\end{verbatim}
\subsubsection{iterate\_convert\_wind()}


\begin{verbatim}
 Submit a wind conversion to job queue.

          Returns the job number of the conversion in the job queue.

          Arguments:
               onshorepowercurve: A powercurve object for use onshore.
               offshorepowercurve: A powercurve object for use offshore.
               onshoremap: If given, use this named .npy file in your server folder
                           as an onshoremap instead of the default CFSR land sea mask.
               nthreads: If given, run the conversion and aggregation
                         with this many threads. Requires privileges to set.
               worksize: If given, each thread converts/aggregates this many
                         hours at a time. Requires privileges to set.

     
          The result is stored as a C array in the server RAM disk.
          To open it, it's a good idea to directly memory map the file
          as an array. The dimensions of the array are length x shape,
          where length and shape are in the meta.json file.
          To get hour 20, for cell 4,1, access result[20][4][1].

          A powercurve object is a dictionary with three keys: HUB_HEIGHT, V and POW.
          The value for HUB_HEIGHT is the hub height in meters above ground.
          The values for V and POW are equal length lists of numbers.
          POW[i] is the power output for a wind speed V[i].
          The V's must be increasing for the linear interpolation routine to work,
          and be in units of meters per second.

          Note: To get production values between 0 and 1, send scaled down power curves!
          The timeseries are simply a weighted sum over production in the entire cutout,
          with the layout cells as weights. The unit of the sum is the unit of the layout
          times the unit on the second axis of the power curve.
          
\end{verbatim}
\subsubsection{iterate\_convert\_pv()}


\begin{verbatim}
 Submit a photovoltaics conversion to job queue.

          Arguments:
               solar_panel_config: A solar panel configuration object,
               nthreads: If given, run the conversion and aggregation
                         with this many threads. Requires privileges to set.
               worksize: If given, each thread converts/aggregates this many
                         hours at a time. Requires privileges to set.

          Returns the job number of the conversion in the job queue.
    
          The result is stored as a C array in the server RAM disk.
          To open it, it's a good idea to directly memory map the file
          as an array. The dimensions of the array are length x shape,
          where length and shape are in the meta.json file.
          To get hour 20, for cell 4,1, access result[20][4][1].

          A solar_panel_config object is a dictionary with the following keys:
               A, B, C: The A,B, and C coefficients in the expression
                    reference_effeciency(I) = A + B(I) + C*log(I)
                    (I is the incident radiation). (See [1])
               D: The temperature power coefficient of the PV cell.
               NOCT: The Normal Operating Cell Temperature (in Kelvin)
               Tstd: The Standard Test Conditions temperature (in Kelvin)
               Tamb: The ambient temperature at STC (in Kelvin)
               Intc: Normal Testing Conditions irradiance (in W/m^2)
               ta: Product of transmittance and absorptance for the glass panel
                   in front of the solar cell. (0.9 is typical)
               threshold: Solar radiation below which the panel stops giving useful power
                          (set to e.g. 1 to avoid division by zero)
               inverter_efficiency: The (constant) efficiency of the inverter.
                                    The entire timeseries is multiplied with this number.

          The input orientations will be reset after a successful call.

          Note: To get production values in units of installed capacity
                (for a single orientation), calculate the A,B and C coefficients
                for 1 m^2 solar panel, and divide the weight by the rated production
                for 1 m^2 of the solar panel.

          The timeseries are simply a weighted sum over production in the entire cutout,
          with the layout cells as weights. The unit of the sum is the unit of the layout
          times the unit of the production.

          The production in each cell is calculated as a sum of the production
          for each orientation times that orientations weight.
          The unit of the production in each cell is proportional to W/m^2
          (depending on A,B and C) times the unit of the orientation weights.

          [1]:  "A robust model for the MPP performance of different types of PV-modules
          applied for the performance check of grid connected systems", 2004,
          by Hans Beyer, Gerd Heilscher and Stefan Bofinger
          
\end{verbatim}
\subsubsection{iterate\_convert\_wind\_and\_pv()}


\begin{verbatim}
 Submit a wind /and/ photovoltaics conversion to job queue.

          Arguments:
               onshorepowercurve: A powercurve object for use onshore.
               offshorepowercurve: A powercurve object for use offshore.
               onshoremap: If given, use this named .npy file in your
                           server folder as an onshoremap instead of
                           the default CFSR land sea mask.
:
               solar_panel_config: A solar panel configuration object,
               nthreads: If given, run the conversion and aggregation
                         with this many threads. Requires privileges to set.
               worksize: If given, each thread converts/aggregates this many
                         hours at a time. Requires privileges to set.

          Returns the job number of the conversion in the job queue.

          The result of this function is exactly the same as
          calling iterate_convert_wind() and iterate_convert_pv() separately,
          except it's slightly faster.
          
\end{verbatim}
\subsubsection{iterate\_save\_production\_sum()}


\begin{verbatim}
 Calculate the sum of production in each grid cell.
          
          Arguments:
               result_name: Name of the result. A npy file with this name
                            will be created in your directory containing
                            the sums.
               result_type: Must be either "wind" or "pv".
                            This selects which result to sum.
          
\end{verbatim}
\subsubsection{iterate\_aggregate\_wind()}


\begin{verbatim}
 Submit a wind aggregation to job queue.

          Returns the job number of the conversion in the job queue.

          Arguments:
               result_name: Name of file for storing the result.
               capacitylayouts: A list of file names in your folder containing
                                capacitylayouts to apply.
                                These should all be numpy (.npy) files.
               nthreads: If given, run the conversion and aggregation
                         with this many threads. Requires privileges to set.
               worksize: If given, each thread converts/aggregates this many
                         hours at a time. Requires privileges to set.

          The result is stored in your folder on the server with the name
          you gave as parameter result_name. It is a numpy array with a number
          of timeseries equal to the number of layouts you specified.
          To get e.g. hour 100 for layout number 2, use result_name[99][1]. 
          
\end{verbatim}
\subsubsection{iterate\_aggregate\_pv()}


\begin{verbatim}
 Submit a PV aggregation to job queue.

          Returns the job number of the conversion in the job queue.

          Arguments:
               result_name: Name of file for storing the result.
               capacitylayouts: A list of file names in your folder containing
                                capacitylayouts to apply.
                                These should all be numpy (.npy) files.
               nthreads: If given, run the conversion and aggregation
                         with this many threads. Requires privileges to set.
               worksize: If given, each thread converts/aggregates this many
                         hours at a time. Requires privileges to set.

          The result is stored in your folder on the server with the name
          you gave as parameter result_name. It is a numpy array with a number
          of timeseries equal to the number of layouts you specified.
          To get e.g. hour 100 for layout number 2, use result_name[99][1]. 
          
\end{verbatim}
\subsubsection{iterate\_aggregate\_wind\_and\_pv()}


\begin{verbatim}
 Submit a wind and PV aggregation to job queue.
          Note: this is faster than aggregating wind and pv separately.

          Returns the job number of the conversion in the job queue.

          Arguments:
               result_name: Name of file for storing the result.
               wind_capacitylayouts:
                                A list of file names in your folder containing
                                capacitylayouts to apply for wind.
                                These should all be numpy (.npy) files.
               pv_capacitylayouts:
                                A list of file names in your folder containing
                                capacitylayouts to apply for PV.
                                These should all be numpy (.npy) files.
               nthreads: If given, run the conversion and aggregation
                         with this many threads. Requires privileges to set.
               worksize: If given, each thread converts/aggregates this many
                         hours at a time. Requires privileges to set.

          The result is stored in your folder on the server with the name
          you gave as parameter result_name. It is a numpy array with a number
          of timeseries equal to the number of layouts you specified.
          To get e.g. hour 100 for wind layout number 2, use
          result_name["wind_result"][99][1], or 
          result_name["PV_result"][99][1] for PV.
          
\end{verbatim}
